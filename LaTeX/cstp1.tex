\documentclass[12pt]{article}

\usepackage[utf8x]{inputenc}
\usepackage{amsmath}
\usepackage{amssymb}
\usepackage{amsfonts}
\usepackage{graphicx}
\usepackage{tikz,tkz-tab}
\usepackage{mathrsfs}
\usepackage{listings}
\usepackage[left=2cm,right=2cm,top=2cm,bottom=2cm]{geometry}
\usepackage{multicol}
\usepackage{float,graphicx}
\usepackage{pstricks,pst-all,pst-plot,pstricks-add,pst-3dplot}
\usepackage{cases}
\usepackage{array}
\usepackage[shortlabels]{enumitem}
\usepackage{eurosym}
\usepackage{bbm}
\usepackage[european, straightvoltages, RPvoltages]{circuitikz}
\usepackage{diagbox}

\definecolor{mygreen}{RGB}{28,172,0} % color values Red, Green, Blue
\definecolor{mylilas}{RGB}{170,55,241}

\title{CS221 - TP1}
\author{Vincent MOUCADEAU - Rémi MAZZONE | 2A}
\date{23/11/2022}


\newenvironment{restoretext}%
    {
     \begin{adjustwidth}{}{\leftmargin}%
    }{\end{adjustwidth}
     }
     
\renewcommand*{\overrightarrow}[1]{\vbox{\halign{##\cr 
  \tiny\rightarrowfill\cr\noalign{\nointerlineskip\vskip1pt} 
  $#1\mskip2mu$\cr}}}


\newcommand{\dvec}[1]{\overrightarrow{#1}} % Commande perso pour vecteurs
\newcommand{\fracvec}[3]{\dfrac{#1}{#2}\dvec{#3}}

     
\newcommand*\Vc[2][1ex]{\Vcaux#2,,\Vcaux{#1}}% arg optionnel = espacement entre coordonnées
\def\Vcaux#1,#2,#3,#4\Vcaux#5{%
    \ensuremath{\left(\vcenter{\baselineskip0pt
    \halign{\hfil\kern.25em$##$\kern.25em\hfil\crcr
        #1\cr\noalign{\vskip#5}#2\cr\noalign{\vskip#5}#3\crcr}%
    }\right)}%
} 

\newcolumntype{R}[1]{>{\raggedleft\arraybackslash }b{#1}}
\newcolumntype{L}[1]{>{\raggedright\arraybackslash }b{#1}}
\newcolumntype{C}[1]{>{\centering\arraybackslash }b{#1}}

\renewcommand{\arraystretch}{1.4}

% -------- PYTHON -------- %
\definecolor{dkgreen}{rgb}{0,0.6,0}
\definecolor{gray}{rgb}{0.5,0.5,0.5}
\definecolor{mauve}{rgb}{0.58,0,0.82}

\lstdefinestyle{languageC}{
    language=C,
    aboveskip=3mm,
    belowskip=3mm,
    basicstyle={\small\ttfamily},
    breaklines=true,%
    keywordstyle=\color{blue},%
    morekeywords=[2]{1}, keywordstyle=[2]{\color{black}},
    identifierstyle=\color{black},%
    stringstyle=\color{mylilas},
    commentstyle=\color{mygreen},%
    showstringspaces=false,%without this there will be a symbol in the places where there is a space
    numbers=left,%
    numberstyle={\tiny \color{black}},% size of the numbers
    numbersep=14pt, % this defines how far the numbers are from the text
    emph=[1]{for,end,break},emphstyle=[1]\color{red}, %some words to emphasi
    frame=single,
    frameround={t}{t}{t}{t},
    xleftmargin=.2\textwidth, xrightmargin=.2\textwidth
}

\lstdefinestyle{pseudoCode}{
    aboveskip=3mm,
    belowskip=3mm,
    basicstyle={\small\ttfamily},
    breaklines=true,
    morekeywords={input, output, pour, si, echanger, fin, alors, faire},
    keywordstyle=\color{blue},%
    morekeywords=[2]{1}, keywordstyle=[2]{\color{red}},
    identifierstyle=\color{black},%
    stringstyle=\color{mylilas},
    commentstyle=\color{mygreen},%
    showstringspaces=false,%without this there will be a symbol in the places where there is a space
    numbers=left,%
    numberstyle={\tiny \color{black}},% size of the numbers
    numbersep=14pt, % this defines how far the numbers are from the text
    emph=[1]{for,end,break},emphstyle=[1]\color{red}, %some words to emphasi
    frame=single,
    frameround={t}{t}{t}{t},
    xleftmargin=.2\textwidth, xrightmargin=.2\textwidth
}

\lstdefinestyle{Makefile}{
    aboveskip=3mm,
    belowskip=3mm,
    basicstyle={\small\ttfamily},
    breaklines=true,
    morekeywords={gcc, rm},
    keywordstyle=\color{blue},%
    morekeywords=[2]{1}, keywordstyle=[2]{\color{red}},
    identifierstyle=\color{black},%
    stringstyle=\color{mylilas},
    commentstyle=\color{mygreen},%
    showstringspaces=false,%without this there will be a symbol in the places where there is a space
    numbers=left,%
    numberstyle={\tiny \color{black}},% size of the numbers
    numbersep=14pt, % this defines how far the numbers are from the text
    emph=[1]{for,end,break},emphstyle=[1]\color{red}, %some words to emphasi
    frame=single,
    frameround={t}{t}{t}{t},
    xleftmargin=.2\textwidth, xrightmargin=.2\textwidth
}

% Default fixed font does not support bold face
\DeclareFixedFont{\ttb}{T1}{txtt}{bx}{n}{12} % for bold
\DeclareFixedFont{\ttm}{T1}{txtt}{m}{n}{12}  % for normal

\renewcommand{\contentsname}{Table des matières}

\begin{document}

\maketitle

\tableofcontents
\newpage

\section{Introduction}
Dans ce TP, maintenant que nous avons pris en main les outils de bases, nous allons passer à l'étude d'un système un peu plus complexe, qui traduit un vrai problème. En effet, nous allons étudier le mouvement d'un pendule (sans frottements) avec deux approches différentes : nous utiliserons Simulink dans la première partie et uniquement Matlab dans la deuxième.

\section{Préparation} 
\subsection{Pseudo code du tri "Bubble sort"} 

\begin{lstlisting}[style=pseudoCode, caption=Pseudo code du tri "Bubble sort" optimisé]
input: int *tab, int n
output: int nb_swaps
nb_swaps = 0
pour i = 0 a n-1 faire
    bool swapped = false
    pour j = 0 a n-i-1 faire
        si tab[j] > tab[j+1] alors
            echanger tab[j] et tab[j+1]
            nb_swaps++
            swapped = true
        fin si
    fin pour
    si swapped == false alors
        retourner nb_swaps
    fin si
fin pour
\end{lstlisting} 

\subsection{Makefile} 
\begin{lstlisting}[style=Makefile, caption=Makefile du projet]
main:main.o fonctions.o 
	gcc -o $@ $^

main.o: main.c fonctions.h 
	gcc -c $<

fonctions.o: fonctions.c fonctions.h 
	gcc -c $<

clean: 
	rm -rf *.o 
	rm -rf main
\end{lstlisting} 

\section{Tri bulle}
\begin{enumerate}
    \item Avec le makefile écrit précédemment, le programme se compile et s'exécute correctement. Le programme affiche bien le contenu du tableau \texttt{tab1}.
    \item Implémentation de la fonction \texttt{swap} en C. On utilise une variable temporaire pour échanger les valeurs des deux variables.
\end{enumerate}
    \begin{lstlisting}[style=languageC, caption=Implémentation de la fonction swap]
void swap(int *a, int *b) {
    int temp = *a;
    *a = *b;
    *b = temp;
}
    \end{lstlisting}
\begin{enumerate}[resume]
    \item Implémentation du tri bulle en C. On utilise la fonction \texttt{swap} pour échanger les valeurs des deux variables. On utilise une variable \texttt{swapped} pour savoir si un échange a eu lieu. Si aucun échange n'a eu lieu, on peut arrêter le tri.
\end{enumerate}
    \begin{lstlisting}[style=languageC, caption=Implémentation optimisée du tri bulle]
int bubbleSort(int *tab, int n) {
    int nb_swaps = 0;
    int i, j;
    bool swapped;
    for (i = 0; i < n-1; i++) {
        trie = true;
        for (j = 0; j < n-i-1; j++) {
            if (tab[j] > tab[j+1]) {
                swap(&tab[j], &tab[j+1]);
                nb_swaps++;
                trie = false;
            }
        }
        if (trie) {
            break;
        }
    }
    return nb_swaps;
}
    \end{lstlisting}
\begin{enumerate}[resume]
    \item On vérifie que le tri fonctionne correctement à l'aide de la fonction \texttt{compare} qui compare élément par élément le tableau trié avec le tableau de référence. On affiche le nombre d'échanges effectués.
    \item Résultats du tri bulle pour les tableaux donnés :
    \begin{itemize}
        \item \texttt{tab1} : Comparaison OK, 682 échanges
        \item \texttt{tab2} : Comparaison OK, 1216 échanges
        \item \texttt{tab3} : Comparaison OK, 63 échanges
        \item \texttt{ref} : Comparaison OK, 0 échanges
    \end{itemize}
\end{enumerate}

\newpage 
\section{Conclusion}
Lors de ce TP, nous avons pu simuler le comportement d'un circuit RC + R//C à l'aide de Matlab. Nous avons pu faire une analyse temporelle et fréquentielle du circuit. Nous avons pu observer que la tension $U_c$ est déphasée par rapport à la tension $U_s$.

\end{document}
